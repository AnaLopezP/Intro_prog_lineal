\documentclass[a4paper, 10pt]{article}
\title{Introducción a la Programación Lineal}
\author{Ana López}
\begin{document}
    \maketitle
    \part{Ejercicio 1: Establecer las ecuaciones}
    De los guerreros, los más valiosos son los jinetes, luego los arqueros y luego los espadachines. 
    Cada jinete gasta 140/1200 de comida, 0/800 de madera, 100/600 de oro (+230 poder)
    Cada arquero gasta 80/1200 de comida, 10/800 de madera, 40/600 de oro (+95 poder)
    Cada Espadachín gasta 60/1200 de comida, 20/800 de madera, 0/600 de oro (+70 poder)

    Las variables van a ser el número de jinetes, arqueros y espadachines. (Las variables son números naturales). Para establecer las restricciones, hacemos una ecuación por recurso indicando su gasto en función del número de guerreros (No se pueden repartir más recursos de los que hay, por lo que los recursos utilizados son igual o menor que el total):

    Comida: $140*Jinete + 80*Arquero + 60*Espadach\'in <= 1200$
    Madera: $10*Arquero + 20*Espadach\'in <= 800$
    Oro: $100*Jinete + 40*Arquero <= 600$
    Estas ecuaciones significan que cada guerrero (jinete, arquero, espadachín) gasta un número de recursos cada uno.

    \part{Ejercicio 2: Utilizar la progrmación lineal}
    Para maximizar el poder del ejército, hacemos una función lineal e indicamos al programa que tiene que tener el valor máximo (Código en el archivo .py):
    máx $230*Jinete + 95*Arquero + 70*Espadach\'in$

    \part{Ejercicio 3: Conclusiones}
    El algoritmo de progrmación lineal sirve para optimizar un valor en función de unas restricciones. Aunque hay otras opciones para maximizar o minimizar funciones, como el ensayo y error o el algoritmo genético, es más fácil usar este método, ya que crear ecuaciones lineales y resolverlas es más sencillo y práctico que probar aleatoriamente posibles soluciones (como el primer ejemplo) o que probar todas y cada una de las posibles opciones (algoritmo genético).
\end{document}